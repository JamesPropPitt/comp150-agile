% Please do not change the document class
\documentclass{scrartcl}

% Please do not change these packages
\usepackage[hidelinks]{hyperref}
\usepackage[none]{hyphenat}
\usepackage{setspace}
\doublespace

% You may add additional packages here
\usepackage{amsmath}

% Please include a clear, concise, and descriptive title
\title{Task board productivity; digital vs physical}

% Please do not change the subtitle
\subtitle{COMP150 - Agile Development Practice}

% Please put your student number in the author field
\author{1601002}

\begin{document}

\maketitle

\abstract{Please include an abstract of at most 100 words (these do not count towards your word count).}

This essay will explore the benefits of task boards, both physical and digital, in regards to the impact on overall team performance. Understanding this is useful in helping developers ascertain the appropriate type of task board for their project. This is directly linked with Agile philosophy which favours customer interaction and adaptability over tools and planning.
The Agile philosophy can be implemented and used for game development within a scrum team. This involves many different features of a scrum team being implemented from the Agile methodology, including the tool: task boards. Discovering the relevant form of task board for the developers could be utile in maximising efficiency, which all game developers strive to do.

\section{Introduction}

Meeting to discuss and exchange ideas is core in all business models across the world, particularly those which work on developing products. For a team employing Agile development methods, like scrum, the act of meeting is an important factor for the cohort to function efficiently.
When used in the games industry frequent meetings and updates between departments is crucial for any business, some of these can be in the form of videoconferences, phone calls and task boards.
Task boards are useful for making sure all development departments are on the appropriate path for their team, essentially preventing two teams working on the same task\cite{esbensen2015dboard}. The potential problem with daily meetings to exchange what each member of the team is doing is that the meetings can cover details which are not necessarily relevant. By creating a focal point for the meeting, such as standing up, people will want to share their tasks and progress as briefly as possible as no one enjoys standing up for prolonged periods of time. This is where a digital task board would suffer since people are presumed to be sitting down when they are operating their computer.

\section{Task boards}

In a large scale survey of 3501 participants it was recorded that 73 percent of people use Scrum or a Scrum variant. Due to this statistic and also because of the presence of task boards in all scrum teams\cite{dullemond2014collaboration}, a large portion of people will be looking for ways on how to use them efficiently. A task board is typically situated in the center of a room with sticky notes to represent the various tasks in the assignment which creates a focal point for daily stand-up meetings. Daily stand-up meetings with a task board in face-to-face interaction is not only productive and helps boost relations between co-workers but also agrees with the Agile manifesto\cite{rubart:2014cooperative}. However when members of the team cannot be present in the same room at the same times of a day a virtual/digital task board may need to be used. 


Without a physical task board in front of the attendants at the meeting, some people may lose track of what they have achieved and what they have yet to achieve. Sometimes members could potentially come up with a new plan mid-way through a sprint without officially tracking it.
Digital task boards also have the capability to display when a task was completed, who it was completed by and how quickly it was completed. This can help team leaders get an idea of how their team operates and who works best on what kind of project, something which is not so easy to view with physical task boards.\cite{hajratwala2012task}

\section{Intimidation and motivation}

There are various factors to take into account when making a task board, particularly the personality of the team you will be working with and the skillset that each member brings to the table
If the team is less socially outgoing then perhaps a physical task board may create boundaries between co-workers rather than build on them. However, if the team are technophobes then perhaps steering clear of a digital task board would be wise. 
This is especially true if the digital board becomes oversaturated with information, this could also create an apathetic approach from the members of the team as they are more likely to simply redirect a question to the sprint board online.\cite{perry2008drifting}

\section{Conclusion}

Discussing ideas that each member has come up with is undeniably important in all business models on a global scale. Having discussed and weighed up the advantages and disadvantages of each task board in some detail, the best conclusion to come to is for neither to be used instead of the other, but rather alongside each other. A physical task board without a digital task board can create problems for teams that have trouble meeting regularly, possibly due to distance or if some members are ill. A digital task board without a physical task board can create apathy and might cause confusion for members of the Scrum. Therefore digital task boards are best implemented when used in conjunction with physical task boards and vice versa.

\bibliographystyle{ieeetran}
\bibliography{references}

@article{dullemond2014collaboration,
  title={Collaboration Spaces for Virtual Software Teams},
  author={Dullemond, Kevin and van Gameren, Ben and van Solingen, Rini},
  journal={IEEE Software},
  volume={31},
  number={6},
  pages={47--53},
  year={2014},
  publisher={IEEE}
}

\end{document}